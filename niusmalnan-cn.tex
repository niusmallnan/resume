\documentclass[12pt,a4paper]{moderncv}

% moderncv themes
%\moderncvtheme[blue]{casual}                 % optional argument are 'blue' (default), 'orange', 'red', 'green', 'grey' and 'roman' (for roman fonts, instead of sans serif fonts)
\moderncvtheme[blue]{classic}                % idem
\usepackage{xunicode, xltxtra}
\XeTeXlinebreaklocale "zh"
\widowpenalty=10000

%\setmainfont[Mapping=tex-text]{文泉驿正黑}

% character encoding
%\usepackage[utf8]{inputenc}                   % replace by the encoding you are using
\usepackage{CJKutf8}

%\usepackage[colorlinks,linkcolor=red,anchorcolor=blue,citecolor=green]{hyperref}
  
% adjust the page margins
\usepackage[scale=0.8]{geometry}
\recomputelengths                             % required when changes are made to page layout lengths
\setmainfont[Mapping=tex-text]{Hiragino Sans GB W3}
\setsansfont[Mapping=tex-text]{Hiragino Sans GB W3}
\CJKtilde

% personal data
\firstname{张智博}
\familyname{}
\title{@牛小腩}               % optional, remove the line if not wanted
\homepage{github.com/niusmallnan}
\mobile{13655713575}                    % optional, remove the line if not wanted
\email{zhangzhibo521@gmail.com}                      % optional, remove the line if not wanted
\quote{\small{树叶绿了又黄还是穷得瑟}}

\nopagenumbers{}

\begin{document}

\maketitle

\section{工作经历}
\renewcommand{\baselinestretch}{1.2}

\cventry{2011~至今}
{美食行}
{Python, Android}
{创业项目}{}
{美食行是一个分享记录发现美食的社区,通过别人的分享,帮助用户找到喜欢的美食及餐馆。我在团队中主要负责全部的AndroidApp的开发工作以及部分服务端开发工作。{}%
\begin{itemize}%
\item AndroidApp大部分使用原生Java开发,期间完成4次大的UI变动,实现了包括瀑布流、推拉门等效果,较完善的客户端cache系统,图片加载防OOM机制,详情请参考客户端下载(\href{http://meishixing.com/files/meishixing_last.apk}{猛击此处})
\item 使用tornado应用服务器,Redis数据库作为CacheDB,MongoDB进行地理检索
\item 修改了tornado部分源码,给其增加了一个模版自动重载的功能
\item 基于thrift和redis实现的session center
\item 基于django实现的微博时光机
\end{itemize}}

\vspace*{\baselineskip}
\cventry{2011}
{阿里云}
{Android, Javascript}
{云OS前端}{}
{口碑网被分拆到阿里云,遂转岗到云os前端部门,从事云应用相关开发,由于公司的剧烈动荡,业务不稳定,这段时间历练明显不足。{}%
\begin{itemize}%
\item 基于processing.js实现的画图云应用
\end{itemize}}

\vspace*{\baselineskip}
\cventry{2009--2011}
{口碑网}
{Java, Javascript}
{地图搜索}{}
{大四实习期以及第一份工作,均在口碑网本地搜索部门任职,主要负责口碑网地图搜索业务的开发工作。这份工作兼具前后端,包括后端的数据api服务以及前端地图的js(YUI)交互逻辑,使我得到了很基础的双向锻炼。{}%
\begin{itemize}%
\item 基于nodejs实现长连接服务,并应用于实际业务。\href{http://www.mhtml5.com/events/hzsl6}{《使用nodejs打造comet系统》}
\item 基于地图api和Nginx的cache组件,实现的地图瓦片多点渲染功能
\item 基于JSPtag,实现的静态资源文件加载,使得各调用方无需关心版本号
\item 基于Diffable技术,实现静态资源的差量加载功能
\end{itemize}}

\section{个人简介}
\cvitem{}
{牛小腩者,本姓张,名智博,生于民风彪悍之辽地,其民喜歌舞,暮夜男女群聚,相就歌戏。腩少时“威武”,素丰硕,不能剧步,故得“小胖”之名号。其父母虽为常人,但对其儿品行教之甚严,腩常曰:少时父母之教育,吾感激涕零。}
\cvitem{}
{后闻“上有天堂,下有苏杭”,于弱冠之年游学于杭州,虽天资鲁钝,仍刻苦研习计算机之科学,喜读史书,嗜好广泛。而今,学业即成,身为男儿,当以民族大业为己任,为党国大业而奋进,腩恨其未能从军,对其未能为党国尽忠而抱终生之憾。思前想后,转战IT,望可抗蛮夷,并击之,以振民族之魂魄,国之声威。}
\cvitem{}
{初出校门,运甚佳,得进阿里,技术研发部谋一职位,奋勇争先,一年有余,升职加薪以奖励。二年,公司动荡,员工纷纷出走,腩遂加入创业公司共赴大业,奋发二年,前景暗淡,然在外漂泊已久,思乡之情愈发强烈,辞职回家,望全忠厚礼仪,孝顺父母,关爱家人。}
\cvitem{}
{to be continue...}

\section{工作技能}
\cventry{语言}{Python = Java(J2EE \& Android) > Javascript > Others}{}{}{}{}
\cventry{数据库}{Mysql > Redis > MongoDB}{}{}{}{}
\cventry{操作系统}{MacOS = Ubuntu > Windows}{}{}{}{}
\cventry{IDE}{Eclipse = Vim}{}{}{}{}
\cventry{代码管理}{git > svn}{}{}{}{}

\section{教育}
\cventry{2006--2010}{本科}{浙江理工大学 管理科学与工程}{}{}{}

\section{兴趣爱好}
\cventry{闲情逸致}{douban}{吟诗作对人生快事,部分作品链接 (\href{http://www.douban.com/people/niusmallnan/notes?start=30}{猛击此处})}{}{}{}
\cventry{朱弦三叹}{music}{classic \& rock \& folk}{}{}{}
\cventry{偷师学艺}{github}{fork能够帮助业务的项目是一大快事}{}{}{}
\cventry{游戏人生}{WoW}{我只是个低调的90级熊猫人小FS,半AFK状态}{}{}{}


\renewcommand{\baselinestretch}{1.0}

\closesection{}                   % needed to renewcommands
\renewcommand{\listitemsymbol}{-} % change the symbol for lists

\end{document}
